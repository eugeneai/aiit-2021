\documentclass[conference]{IEEEtran}
\IEEEoverridecommandlockouts
% The preceding line is only needed to identify funding in the first footnote. If that is unneeded, please comment it out.
\usepackage{cite}
\usepackage{amsmath,amssymb,amsfonts}
\usepackage{algorithmic}
\usepackage{graphicx}
\usepackage{textcomp}
\usepackage{xcolor}
\def\BibTeX{{\rm B\kern-.05em{\sc i\kern-.025em b}\kern-.08em
    T\kern-.1667em\lower.7ex\hbox{E}\kern-.125emX}}

\usepackage{luatextra}
\usepackage{currvita}
%
\usepackage{makeidx}  % allows for indexgeneration
\usepackage{url}
%\usepackage[biblabel]{cite}
%\usepackage{amsmath,amssymb,amsfonts}
%\usepackage{algorithmic}
%\usepackage{graphicx}
%\usepackage[pdftex]{graphicx}
%\usepackage[dvips]{graphicx}
\usepackage{alltt}
%\usepackage{textcomp}
\usepackage{enumitem}
%\usepackage{indentfirst}

%
% Used for displaying a sample figure. If possible, figure files should
% be included in EPS format.
%
% If you use the hyperref package, please uncomment the following line
% to display URLs in blue roman font according to Springer's eBook style:
% \renewcommand\UrlFont{\color{blue}\rmfamily}


\usepackage{hyperref}
\hypersetup{
    % bookmarks=true,         % show bookmarks bar?
    unicode=true,          % non-Latin characters in Acrobat’s bookmarks
    pdftoolbar=true,        % show Acrobat’s toolbar?
    pdfmenubar=true,        % show Acrobat’s menu?
    pdffitwindow=false,     % window fit to page when opened
    pdfstartview={FitH},    % fits the width of the page to the window
    %pdftitle={},    % title
    %pdfauthor={Author},     % author
    %pdfsubject={Subject},   % subject of the document
    %pdfcreator={Creator},   % creator of the document
    %pdfproducer={Producer}, % producer of the document
    %pdfkeywords={keyword1, key2, key3}, % list of keywords
    %pdfnewwindow=true,      % links in new PDF window
    colorlinks=true,       % false: boxed links; true: colored links
    linkcolor=black,          % color of internal links (change box color with linkbordercolor)
    citecolor=black,        % color of links to bibliography
    filecolor=black,      % color of file links
    urlcolor=black,           % color of external links
    final=true
  }

\setmainfont{Times New Roman}

\begin{document}

\title{Recommender system for high school entrants having social network accounts}

\author{%
\IEEEauthorblockN{Viktoria Kopylova${}^1$, N. Luk'yranov${}^1$, Evgeny Cherkashin${}^{2,1}$}
\IEEEauthorblockA{${}^1$~\textit{Institute for computer science and data analysis,
National research Irkutsk state technical university} Irkutsk, Russia\\
\{,\}@istu.edu}
\IEEEauthorblockA{${}^2$~\textit{Laboratory of Complex informational systems, Institute for system dynamics and control theory, SB RAS,}
Irkutsk, Russia \\
eugeneai@irnok.net}
}

\maketitle

\begin{abstract}
  The research of the paper is devoted to the development a recommender system for entrants choosing the specialty of a high school.  The recommendation is to be realized on the base of data obtained from his/her account of a social network.  Recommenter system data analysis engine is a complex of data analysis technologies being used for social network data extraction and storage, enrolling documents parsing, entrants and study subjects classifications, construction of the correlations between entrants and subjects of study.

  This paper deals with mostly with the stage of information acquisition, correlation patterns construction and taxonomy construction for entrants and high school subjects.  The future development is indicated.
\end{abstract}

\begin{IEEEkeywords}
recommender system, correlation patterns, taxonomy
\end{IEEEkeywords}

\section{Introduction}

Each school student sooner or later faces the problem of choice of a future specialty, university to enroll, and the set of courses to study.  Decisions made at this time significantly affect the future life of the students, their job and career perspective, families' life level, \emph{etc}.  Therefore, any help at this stage seems to be a necessary moment. In comparison to the real estate, where there is a whole industry supporting decision-making -- the services of realtors --, in this field the educational environment supports only an information provision service: students are given only career guidance, advertisement of the high school capabilities.  The student is on his/her own in the fundamental life decision.  Of cause, parents and favorite mentors can supply the additional life experience examples helping to focus the analysis of the present conditions, but the actuality of their advices are generally of questions due to specifics of their experience, subjectivity, absence of the actual information of the present state of the labor market, scientific progress, \emph{etc}.

[[Here we speak about actuality... how to obtain it with data analysis relating contemporary enrollments and, possibly, the resulting educational capabilities of the students after the graduation.]] [[Then transit to the preliminary part (at this time we will deal with the entrant stage) - choice of the study courses]]

Sometimes, having received an education, people realize that they do not want to work in this area. [[In order to minimize the percentage of such situations, educational institutions hold various events: open days, career guidance work among school graduates, posting useful information on the websites of their university or college, \emph{etc}.  These methods help educational institutions in finding and attracting potential applicants for admission and further study in a more efficient and purposeful way.] NOW SEEMS REDUNDAND]

In a period of intense competition, multidisciplinary educational institutions with similar areas of study are interested in a tool that allows them to determine the target audience more accurately, as well as entrants are interested to find a relevant university.  The work with the audience would be focused and directed. This will help to attract applicants interested in a particular direction, which is being implemented in the educational institution.  This tool could improve the effectiveness of career guidance activities for high school students.  Recommender systems are the tools of such kind.

Recommender systems (RS) \cite{rs_basics} are decision support information systems designed to assess the user's level of interest in a particular product or service (object) based on available information about user and object.  The RS development industry began to actively develop with the emergence of online sales services, and now it is one of the active areas of development of decision support systems, a direction of artificial intelligence, focused primarily on commercial use, as well as on solving problems of increasing the productivity of searching for relevant information.

Functioning of a RS could be based on analysis of entrants' personal properties, matching them with already enrolled students, analyzing existing individual direction of training and even a particular educational program.  We have been developing and implementing a prototype RS for helping to choose a direction of training at an university for entrants on the base of algorithms for predicting their possible choice based on digital footprints in a social network. As a source data, \emph{e.g.}, a subscriptions of Russian social network ``In Contact'' (IC) (in Russian -- ``В контакте'') are used as sources of personal characteristics, and enrollment orders of our university.  %The research \cite{c1} of Tomsk state university, Russia, shows that students (pupils) once or twice a year actualize their profile and subscriptions, as irrelevant information in the main tape (blog) results in necessity to routinely filtering it out. In 2016,  the university researchers showed the IC profile in general actually reflects the cognitive properties of personality, and the educational interests of students are closely correlate to their social network behavior.

Nowadays social networks are on of the main place in the life of a person, and even more so a teenager.  This is clear from the data from the Glas Runet online poll service.  Among the 2000 participants of the survey, the majority (86\%) of them know about the existence of Internet social networks and use their services.  Among those who know about the existence of social networks, only 10\% do not use them \cite{c2}.
The most popular social networks according to the Romir holding are Odnoklassniki, Vkontakte and Instagram.  The youngest audience, from 14 to 30 years old, visit Vkontakte, 45\% does it daily, and 70\% does more often than once a day \cite{c6}.

Often, it is in social networks that one can track the self-expression, interests and attitude to the world of a person. The analysis of groups, posts on the main blog, reposts, statuses, \emph{etc.} can be the sources of the valuable information \cite{c7}.

Of course, if a person is an adult, then it is unlikely that all subscriptions to groups will be relevant (``entered and forgot'').  But school students, according to research \cite{11}, update subscriptions every six months or a year: irrelevant content in the feed annoys them.  Accordingly, subscriptions can become the original dataset that can be analyzed.

In 2016, TSU proved that the Vkontakte profile reflects the cognitive personality traits, \emph{i.e.} the educational interests of schoolchildren are closely related to their behavior in the social network \cite{c9}.

%Based on the data on students enrolled in various areas of an university, it is possible to build a system of recommendations for the direction of training.

\section{General approach}


\section{Technologies used}
\label{sec:tech}

At this stage we decided to use C\# based platform for implementation of the RS.  The RS is a web ASP.NET application with database represented with Entity Framework.

\subsection{Input data acquisition}
\label{sec:tech-input}

The input data is provided by the input forms and automatic processing of students enrollment orders in DOCX format, using our syntactic analyzer.  Orders contain header, which is parsed to determine the date of the enrollment, specialty and other parameters.  The only table part lists of the enrolled students.  Additionally the list records are checked by regular expressions determining the sum of points gained.  The reading of the document data is implemented by means of C\# library \texttt{Microsoft.Office.Interop.Word} with its \texttt{Paragraphs} collection and \texttt{Range} objects.

The list of students is traversed, and for each student its IC profile is queried.  The data of interest is the profile and the subscriptions.  The query is done via IC API as HTTP queries.  According to IC API policy, the queries can be issued not more than three times per second, otherwise error 6 meaning ``Too many requests per second'' is obtained.  Additional constraints are stated to the set of issued remote methods.  The policy for the set is not published, its violation results to the Captcha test issue or rejection of a concrete method usage with no restriction for others.

% The database is used to query stored data

One student can have more than one IC accounts.  In this case a form appeared to choose the right one manually.  The decision is stored in the database tale attribute.


\section{Preliminary data processing}
\label{sec:relim-proc}

The acquired input data stored in the database at first is processed with hierarchical cluster analysis algorithms to construct set of classes for students and study courses.  After that an classification procedure is applied to construct relations between individual student to its class.  These classifications allow us to reduce the complexity of the main recommender algorithms and shift partially from collaborative filtration technique to .


\section{Recommendations generation}
\label{sec:proc-recs}


\section{Discussion}
\label{sec:disc}

\section*{Conclusion}
\label{sec:conc}



\section*{Acknowledgements}
\label{sec:ack}



\subsection{Maintaining the Integrity of the Specifications}

The IEEEtran class file is used to format your paper and style the text. All margins,
column widths, line spaces, and text fonts are prescribed; please do not
alter them. You may note peculiarities. For example, the head margin
measures proportionately more than is customary. This measurement
and others are deliberate, using specifications that anticipate your paper
as one part of the entire proceedings, and not as an independent document.
Please do not revise any of the current designations.

\section{Prepare Your Paper Before Styling}
Before you begin to format your paper, first write and save the content as a
separate text file. Complete all content and organizational editing before
formatting. Please note sections \ref{AA}--\ref{SCM} below for more information on
proofreading, spelling and grammar.

Keep your text and graphic files separate until after the text has been
formatted and styled. Do not number text heads---{\LaTeX} will do that
for you.

\subsection{Abbreviations and Acronyms}\label{AA}
Define abbreviations and acronyms the first time they are used in the text,
even after they have been defined in the abstract. Abbreviations such as
IEEE, SI, MKS, CGS, ac, dc, and rms do not have to be defined. Do not use
abbreviations in the title or heads unless they are unavoidable.

\subsection{Units}
\begin{itemize}
\item Use either SI (MKS) or CGS as primary units. (SI units are encouraged.) English units may be used as secondary units (in parentheses). An exception would be the use of English units as identifiers in trade, such as ``3.5-inch disk drive''.
\item Avoid combining SI and CGS units, such as current in amperes and magnetic field in oersteds. This often leads to confusion because equations do not balance dimensionally. If you must use mixed units, clearly state the units for each quantity that you use in an equation.
\item Do not mix complete spellings and abbreviations of units: ``Wb/m\textsuperscript{2}'' or ``webers per square meter'', not ``webers/m\textsuperscript{2}''. Spell out units when they appear in text: ``. . . a few henries'', not ``. . . a few H''.
\item Use a zero before decimal points: ``0.25'', not ``.25''. Use ``cm\textsuperscript{3}'', not ``cc''.)
\end{itemize}

\subsection{Equations}
Number equations consecutively. To make your
equations more compact, you may use the solidus (~/~), the exp function, or
appropriate exponents. Italicize Roman symbols for quantities and variables,
but not Greek symbols. Use a long dash rather than a hyphen for a minus
sign. Punctuate equations with commas or periods when they are part of a
sentence, as in:
\begin{equation}
a+b=\gamma\label{eq}
\end{equation}

Be sure that the
symbols in your equation have been defined before or immediately following
the equation. Use ``\eqref{eq}'', not ``Eq.~\eqref{eq}'' or ``equation \eqref{eq}'', except at
the beginning of a sentence: ``Equation \eqref{eq} is . . .''

\subsection{\LaTeX-Specific Advice}

Please use ``soft'' (e.g., \verb|\eqref{Eq}|) cross references instead
of ``hard'' references (e.g., \verb|(1)|). That will make it possible
to combine sections, add equations, or change the order of figures or
citations without having to go through the file line by line.

Please don't use the \verb|{eqnarray}| equation environment. Use
\verb|{align}| or \verb|{IEEEeqnarray}| instead. The \verb|{eqnarray}|
environment leaves unsightly spaces around relation symbols.

Please note that the \verb|{subequations}| environment in {\LaTeX}
will increment the main equation counter even when there are no
equation numbers displayed. If you forget that, you might write an
article in which the equation numbers skip from (17) to (20), causing
the copy editors to wonder if you've discovered a new method of
counting.

{\BibTeX} does not work by magic. It doesn't get the bibliographic
data from thin air but from .bib files. If you use {\BibTeX} to produce a
bibliography you must send the .bib files.

{\LaTeX} can't read your mind. If you assign the same label to a
subsubsection and a table, you might find that Table I has been cross
referenced as Table IV-B3.

{\LaTeX} does not have precognitive abilities. If you put a
\verb|\label| command before the command that updates the counter it's
supposed to be using, the label will pick up the last counter to be
cross referenced instead. In particular, a \verb|\label| command
should not go before the caption of a figure or a table.

Do not use \verb|\nonumber| inside the \verb|{array}| environment. It
will not stop equation numbers inside \verb|{array}| (there won't be
any anyway) and it might stop a wanted equation number in the
surrounding equation.

\subsection{Some Common Mistakes}\label{SCM}
\begin{itemize}
\item The word ``data'' is plural, not singular.
\item The subscript for the permeability of vacuum $\mu_{0}$, and other common scientific constants, is zero with subscript formatting, not a lowercase letter ``o''.
\item In American English, commas, semicolons, periods, question and exclamation marks are located within quotation marks only when a complete thought or name is cited, such as a title or full quotation. When quotation marks are used, instead of a bold or italic typeface, to highlight a word or phrase, punctuation should appear outside of the quotation marks. A parenthetical phrase or statement at the end of a sentence is punctuated outside of the closing parenthesis (like this). (A parenthetical sentence is punctuated within the parentheses.)
\item A graph within a graph is an ``inset'', not an ``insert''. The word alternatively is preferred to the word ``alternately'' (unless you really mean something that alternates).
\item Do not use the word ``essentially'' to mean ``approximately'' or ``effectively''.
\item In your paper title, if the words ``that uses'' can accurately replace the word ``using'', capitalize the ``u''; if not, keep using lower-cased.
\item Be aware of the different meanings of the homophones ``affect'' and ``effect'', ``complement'' and ``compliment'', ``discreet'' and ``discrete'', ``principal'' and ``principle''.
\item Do not confuse ``imply'' and ``infer''.
\item The prefix ``non'' is not a word; it should be joined to the word it modifies, usually without a hyphen.
\item There is no period after the ``et'' in the Latin abbreviation ``et al.''.
\item The abbreviation ``i.e.'' means ``that is'', and the abbreviation ``e.g.'' means ``for example''.
\end{itemize}
An excellent style manual for science writers is \cite{b7}.

\subsection{Authors and Affiliations}
\textbf{The class file is designed for, but not limited to, six authors.} A
minimum of one author is required for all conference articles. Author names
should be listed starting from left to right and then moving down to the
next line. This is the author sequence that will be used in future citations
and by indexing services. Names should not be listed in columns nor group by
affiliation. Please keep your affiliations as succinct as possible (for
example, do not differentiate among departments of the same organization).

\subsection{Identify the Headings}
Headings, or heads, are organizational devices that guide the reader through
your paper. There are two types: component heads and text heads.

Component heads identify the different components of your paper and are not
topically subordinate to each other. Examples include Acknowledgments and
References and, for these, the correct style to use is ``Heading 5''. Use
``figure caption'' for your Figure captions, and ``table head'' for your
table title. Run-in heads, such as ``Abstract'', will require you to apply a
style (in this case, italic) in addition to the style provided by the drop
down menu to differentiate the head from the text.

Text heads organize the topics on a relational, hierarchical basis. For
example, the paper title is the primary text head because all subsequent
material relates and elaborates on this one topic. If there are two or more
sub-topics, the next level head (uppercase Roman numerals) should be used
and, conversely, if there are not at least two sub-topics, then no subheads
should be introduced.

\subsection{Figures and Tables}
\paragraph{Positioning Figures and Tables} Place figures and tables at the top and
bottom of columns. Avoid placing them in the middle of columns. Large
figures and tables may span across both columns. Figure captions should be
below the figures; table heads should appear above the tables. Insert
figures and tables after they are cited in the text. Use the abbreviation
``Fig.~\ref{fig}'', even at the beginning of a sentence.

\begin{table}[htbp]
\caption{Table Type Styles}
\begin{center}
\begin{tabular}{|c|c|c|c|}
\hline
\textbf{Table}&\multicolumn{3}{|c|}{\textbf{Table Column Head}} \\
\cline{2-4}
\textbf{Head} & \textbf{\textit{Table column subhead}}& \textbf{\textit{Subhead}}& \textbf{\textit{Subhead}} \\
\hline
copy& More table copy$^{\mathrm{a}}$& &  \\
\hline
\multicolumn{4}{l}{$^{\mathrm{a}}$Sample of a Table footnote.}
\end{tabular}
\label{tab1}
\end{center}
\end{table}

\begin{figure}[htbp]
%\centerline{\includegraphics{fig1.png}}
\caption{Example of a figure caption.}
\label{fig}
\end{figure}

Figure Labels: Use 8 point Times New Roman for Figure labels. Use words
rather than symbols or abbreviations when writing Figure axis labels to
avoid confusing the reader. As an example, write the quantity
``Magnetization'', or ``Magnetization, M'', not just ``M''. If including
units in the label, present them within parentheses. Do not label axes only
with units. In the example, write ``Magnetization (A/m)'' or ``Magnetization
\{A[m(1)]\}'', not just ``A/m''. Do not label axes with a ratio of
quantities and units. For example, write ``Temperature (K)'', not
``Temperature/K''.

\section*{Acknowledgment}

The preferred spelling of the word ``acknowledgment'' in America is without
an ``e'' after the ``g''. Avoid the stilted expression ``one of us (R. B.
G.) thanks $\ldots$''. Instead, try ``R. B. G. thanks$\ldots$''. Put sponsor
acknowledgments in the unnumbered footnote on the first page.

\section*{References}

Please number citations consecutively within brackets \cite{IEEEhowto:IEEEtranpage}. The
sentence punctuation follows the bracket \cite{b2}. Refer simply to the reference
number, as in \cite{b3}---do not use ``Ref. \cite{b3}'' or ``reference \cite{b3}'' except at
the beginning of a sentence: ``Reference \cite{b3} was the first $\ldots$''

Number footnotes separately in superscripts. Place the actual footnote at
the bottom of the column in which it was cited. Do not put footnotes in the
abstract or reference list. Use letters for table footnotes.

Unless there are six authors or more give all authors' names; do not use
``et al.''. Papers that have not been published, even if they have been
submitted for publication, should be cited as ``unpublished'' \cite{b4}. Papers
that have been accepted for publication should be cited as ``in press'' \cite{b5}.
Capitalize only the first word in a paper title, except for proper nouns and
element symbols.

For papers published in translation journals, please give the English
citation first, followed by the original foreign-language citation \cite{b6}.

%\bibliographystyle{./bibliography/IEEEtran}
%\bibliography{./bibliography/IEEEabrv,./bibliography/IEEEexample}

\begin{thebibliography}{99}
  \bibitem{rs_basics} D. Jannach, M. Zanker, A. Felfernig, G. Friedrich. Recommender crSystems: An Introduction. Cambridge University Press (2010).
\end{thebibliography}

\vspace{12pt}
\color{red}
IEEE conference templates contain guidance text for composing and formatting conference papers. Please ensure that all template text is removed from your conference paper prior to submission to the conference. Failure to remove the template text from your paper may result in your paper not being published.

\end{document}

%%% Local Variables:
%%% mode: latex
%%% TeX-master: t
%%% End:
